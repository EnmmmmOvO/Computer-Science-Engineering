\documentclass{article}

\usepackage{hyperref}
\usepackage{mathtools}
\usepackage{amssymb}
\usepackage{amsthm}
\usepackage{enumitem}
\usepackage[utf8]{inputenc}

\newcommand{\set}[1]{\{#1\}}
\DeclareMathOperator{\nat}{\mathbb{N}}

\begin{document}
\title{Homework 1 Sample Answers \\ COMP3151 T2 2022}
\author{Vincent Jackson \\ Raphael Douglas Giles \\ Johannes {\AA}man Pohjola}
\maketitle

\section{Question 1: Circularity}

Leslie Lamport's Pants.

\section {Question 2: Dining Cryptographers}

\paragraph{Functional correctness}
If the NSA paid, the new protocol behaves exactly as the original.

If the NSA did not pay, suppose without loss of generality that
cryptographer $1$ paid, and let $k_i$ be a boolean which is true
if the coin flipped by cryptographer $i$ is heads.

In the old protocol, the parity of all the announcements
is given by the following expression:
\[
\neg(k_1 \oplus k_3) \oplus (k_1 \oplus k_2) \oplus (k_2 \oplus k_3) 
\]
\noindent In the new protocol, it's instead given by:
\[
(k_1 \oplus k_3) \oplus (\neg k_1 \oplus k_2) \oplus (k_2 \oplus k_3)
\]
These are equivalent, because $\neg(p \oplus q) = \neg p \oplus q$.

\paragraph{Confidentiality}
Confidentiality is preserved.

Suppose that someone paid, and let's consider the
situation from the perspective of cryptographer 2, who is not paying.
Let the bit $a_i$ denote the equals/different announcement by
cryptographer $i$.
We thus have the following givens: $a_1,a_2,a_3$ and $k_2$ (our own coin toss).
From these we can infer the value $k_3$.\footnote{Since cryptographers do not lie
in their public announcemements, $a_3 = k_2 \oplus k_3$ must hold.}

The only information we're missing is $k_1$ (because we don't know whether
cryptographer 1 lied about their coin toss result or not).
By flipping this bit, we can change from scenarios where cryptographer 1
paid to scenarios where cryptographer 3 paid, without changing any of
the givens.
Hence, the observations made by cryptographer 2 are consistent with either
1 or 3 being the payer.

\section{Question 3}

\subsection{Limit closures}

There are many possible answers. Here's two:

\paragraph{Basic answer}
Let $t$ be some state such that $s \neq t$.
We can then define

\[A = \{\sigma : \exists n\in\mathbb{N}.\;\sigma = s^nt^\omega\}\]

In words: $\sigma \in A$ holds iff $\sigma$ comprises
finitely many copies of $s$, followed by infinite repetition of $t$.

\paragraph{Cheeky answer}
We could also go with this:

\[A = \Sigma^\omega - \{s^\omega\}\]

\subsection{Alpern and Schneider's Theorem}

\begin{enumerate}
  \item {
    For this question, you were expected to use Alpern and Schneider's
    separation theorem to determine the safety and liveness components of the property
    $$P = \{ \sigma \mid \sigma\ \text{contains one `b'} \}.$$

    Recall that the AS separation theorem gives us the following result:
    For every property $P$,
      $$P = \bar{P} \cap (\Sigma^\omega \setminus (\bar{P} \setminus P)),$$
    $\bar{P}$ is a safety property, and
    $\Sigma^\omega \setminus (\bar{P} \setminus P)$ is a liveness property,
    where $\bar{P}$ is the limit closure of $P$.

    \paragraph{Safety} The safety property is
    $$\bar{P} = \{ \sigma \mid \sigma\ \text{contains one `b'} \} \cup \{ a^\omega \},$$
    that is, the set of behaviours that contain at most one `b'.

    This is because the sequence of words
    \begin{gather*}
      baaaa\dots\\
      abaaa\dots\\
      aabaa\dots\\
      aaaba\dots\\
      \vdots
    \end{gather*}
    all occur in $P$, and converge to $a^\omega$,
    and that is the only such convergent behaviour not already in $P$.

    \paragraph{Liveness} The liveness property is
    \begin{align*}
      &\Sigma^\omega \setminus (\bar{P} \setminus P)) \\
      &= \\
      &\Sigma^\omega \setminus ((\{ \sigma \mid \sigma\ \text{contains one `b'} \}
        \cup \{ a^\omega \}) \setminus \{ \sigma \mid \sigma\ \text{contains one `b'} \})) \\
      &= \\
      &\Sigma^\omega \setminus \{ a^\omega \},
    \end{align*}
    that is, the set of all behaviours which contain at least one `b'.

    N.B. For those students familiar with regular expressions, the above can
    all be rendered much more succinctly by
    $P = a^* b a^\omega$,
    $P_S = a^* b a^\omega + a^\omega$,
    and $P_L = \Sigma^* b \Sigma^\omega$.
  }
  \item {
    \begin{align*}
      &\Sigma^\omega \setminus (\bar{P} \setminus P) \\
      &= (\text{$P$ is safety, thus $\bar{P} = P$}) \\
      &\Sigma^\omega \setminus (P \setminus P)) \\
      &=\\
      &\Sigma^\omega \setminus \emptyset \\
      &=\\
      &\Sigma^\omega.
    \end{align*}
  }
  \item {
    The closure of the empty property $\bar{\emptyset} = \emptyset$ and is not
    $\Sigma^\omega$ (we assume here there is at least one state).
    Thus $\emptyset$ is a safety property, the property that bans anything from
    happening at all, trivially violated by any run whatsoever. Moreover, it is
    not a liveness property since is not dense.
  }
\end{enumerate}

\subsection{Bonus Question}

The answer can be found in the original paper of Alpern and Schneider,
\url{https://doi.org/10.1016/0020-0190(85)90056-0}.

If your UNSW Elsevier access is a bit dodgy like mine is, here's another way to do it:

The significance of the existence of at least $2$ distinct states is this implies there are closed sets which are not dense, making this problem non-trivial. Given this, consider the set of behaviours which eventually repeat phrases, i.e. words $\sigma$ such that there exists an $i \geq 0$ and a word $\alpha$ such that $\sigma|_i = \alpha^{\omega}$. We'll give these the suggestive name "rational behaviours" or $Q$. Similarly we'll call the compliment of this set, i.e. words which never end up just repeating forever, the "irrational behaviours" or $I$. Now, clearly these two sets are dense since they can both contain arbitrary finite prefixes, and by definition they must be disjoint. Now consider, for some behaviour $P$, and noting that the union of a dense set with any set is dense,
\[
\begin{split}
(Q \cup P) \cap (I \cup P) &= (Q \cap I) \cup (Q \cap P) \cup (P \cap I) \cup (P \cap P),\\
&= \varnothing \cup P \cup P,\\
&= P.\\
\end{split}
\]
Hence, any property must be the intersection of two liveness properties. Another way to do this is to take $2$ distinct elements $a$ and $b$, which we can do by assumption, and define $Q$ and $I$ like this: $Q = \{xa^{\omega} | x \in \Sigma^{\ast}\}$ and $I = \{xb^{\omega} | x \in \Sigma^{\ast}\}$ and use the same argument as above.

\section{Temporal Logic}

\subsection{Examples}

As this question asked you to interpret English, there were various sensible
`correct' answers. Here are ours.

Let $d$ denote the dragon being alive, and $h$ the princess being happy.
\begin{enumerate}
  \item {
    ``Once the dragon was slain, the princess lived happily ever after.''
    $$\square(\neg d \Rightarrow \square h)$$
  }
  \item {
    ``The dragon was never slain, but the princess lived happily until she didn't.''
    $$\square d \wedge (h \mathrel{\mathcal{U}} \neg h)$$
  }
  \item {
    ``The dragon was slain at least twice.''

    $$\lozenge(\neg d \wedge \lozenge(d \wedge \lozenge\neg d))$$

    The $d$ is there to ensure the dragon comes back to life between being dead,
    so it isn't just slain once and dead for a long time.
  }
  \item {
    ``The dragon was slain at most once.''

    The easiest solution was just take the negation of the previous problem.
    \begin{align*}
      &\neg \lozenge(\neg d \wedge \lozenge(d \wedge \lozenge\neg d)) \\
      &= \\
      &\square \neg (\neg d \wedge \lozenge(d \wedge \lozenge\neg d)) \\
      &= \\
      &\square (\neg\neg d \vee \neg\lozenge(d \wedge \lozenge\neg d)) \\
      &= \\
      &\square (d \vee \square\neg(d \wedge \lozenge\neg d)) \\
      &= \\
      &\square (d \vee \square(\neg d \vee \neg\lozenge\neg d)) \\
      &= \\
      &\square (d \vee \square(\neg d \vee \square d))
    \end{align*}
  }
  \item {
    ``Whenever the dragon was slain, the princess did not live happily.''

    $$\square(\neg d \Rightarrow \neg h)$$
  }
\end{enumerate}

\subsection{Proof}

\begin{enumerate}
  \item {
    \begin{align*}
    &\quad  \sigma \vDash \square\square\varphi \\
    &\text{iff}\quad \text{(by the semantics of $\square$)} \\
    &\quad  \forall i.\ \sigma|_i \vDash \square\varphi \\
    &\text{iff}\quad \text{(by the semantics of $\square$)} \\
    &\quad  \forall i.\ \forall j.\ (\sigma|_i)|_j \vDash \varphi \\
    &\text{iff}\quad \text{(slicing a slice adds the indicies)} \\
    &\quad  \forall i.\ \forall j.\ \sigma|_{i+j} \vDash \varphi \\
    &\text{iff} \quad \text{(reparametrising by $k=i+j$)} \\
    &\quad  \forall k.\ \sigma|_k \vDash \varphi \\
    &\text{iff}\quad \text{(by the semantics of $\square$)} \\
    &\quad  \sigma \vDash \square\varphi
    \end{align*}
  }
  \item {
    \begin{align*}
    &\quad  \sigma \vDash \lozenge\bigcirc\varphi \\
    &\text{iff}\quad \text{(by the semantics of $\lozenge$)} \\
      &\quad  \exists i.\ \sigma|_i \vDash \bigcirc\varphi \\
    &\text{iff}\quad \text{(by the semantics of $\bigcirc$)} \\
    &\quad  \exists i.\ (\sigma|_i)|_1 \vDash \varphi \\
    &\text{iff}\quad \text{(slicing commutes: $\sigma|_i|_j = \sigma|_j|_i$)} \\
    &\quad  \exists i.\ (\sigma|_1)|_i \vDash \varphi \\
    &\text{iff} \quad \text{(by the semantics of $\lozenge$)} \\
    &\quad  \sigma|_1 \vDash \lozenge\varphi \\
    &\text{iff}\quad \text{(by the semantics of $\bigcirc$)} \\
    &\quad  \sigma \vDash \bigcirc\lozenge\varphi
    \end{align*}
  }
  \item {
    \begin{align*}
    &\quad  \sigma \vDash \square\varphi \\
    &\text{iff}\quad \text{(by the semantics of $\square$)} \\
    &\quad \forall i.\ \sigma|_i \vDash \varphi \\
    &\text{then}\quad \text{(first order logic)} \\
    &\quad  \exists i.\ \sigma|_i \vDash \varphi \\
    &\text{iff}\quad \text{(by the semantics of $\lozenge$)} \\
    &\quad  \sigma \vDash \lozenge\varphi
    \end{align*}
  }
\end{enumerate}


\end{document}
