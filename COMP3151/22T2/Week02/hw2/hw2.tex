\documentclass{article}

\usepackage[printqrbox=false,printhint=false,printanswer=true,printmarkingguide=false,printdraftpaper=false]{unswalgos}

\usepackage{tikz}
\usetikzlibrary{patterns}
\usetikzlibrary{shapes,fit}
\usepackage{tkz-fct}
\usepackage{wrapfig}
\usepackage{subfig}

\usepackage{mathtools}
\usepackage{amssymb}
\usepackage{booktabs,multicol,multirow}
\usepackage{wasysym}
\usepackage{tcolorbox}

\DeclareMathOperator*{\argmax}{arg\,max}
\DeclareMathOperator*{\argmin}{arg\,min}
\DeclareMathOperator{\NAND}{NAND}
\DeclareMathOperator{\AND}{AND}
\DeclareMathOperator{\OR}{OR}
\DeclareMathOperator{\NOT}{NOT}

\usepackage{xspace}

\fancyfoot[L]{\leftmark}
\fancyfoot[R]{\rightmark}

\usepackage{listings}

\definecolor{codebrown}{rgb}{0.8,0.44,0.2}
\definecolor{codegray}{rgb}{0.5,0.5,0.5}
\definecolor{codepurple}{rgb}{0.58,0,0.82}
\definecolor{backcolour}{rgb}{0.95,0.95,0.92}

\lstdefinestyle{mystyle}{
    backgroundcolor=\color{backcolour},   
    commentstyle=\color{codebrown},
    keywordstyle=\color{codepurple},
    numberstyle=\tiny\color{codegray},
    stringstyle=\color{codepurple},
    basicstyle=\ttfamily\footnotesize,
    breakatwhitespace=false,
    numbers=left,
    breaklines=true,                 
    captionpos=b,                    
    keepspaces=true,
    showspaces=false,                
    showstringspaces=false,
    showtabs=false,                  
    tabsize=2,
    xleftmargin=0.035\textwidth
}
\lstset{style=mystyle}

\usepackage{framed}
% This enables new paragraphs without indentation
\usepackage[parfill]{parskip}

\newcommand{\sem}{22T2}
\newcommand{\semester}{Term 2, 2022}
\SubjectNo{COMP3151}
\newcommand{\taskname}{Homework 2}
\Institution{Jinghan Wang, z5286124} % Replace this with your name and zID


\begin{document}
\begin{Question} [\large\textbf {No skip {[4 marks]}}]

\begin{lstlisting}[language=Promela]
  byte x = 0;

  active proctype P() {
    x = 1;
    x = 2;
    x = 3;
  }

  ltl skip_2 { <>(x == 1 until x == 3) }
\end{lstlisting}

Here we would be surprised if the value of $x$ becomes $3$ immediately after being $1$. Yet Spin verifies that the property $skip\_2$ holds.

Explain why this model nonetheless satisfies $skip\_2$. Also, give an LTL formula that better captures the intent stated above.

\begin{answer}
Answer: 
\begin{quote}
  When several processes are executed at the same time, it is possible.\\\\
  First, When a process has completed the assignment of $x=2$, it has not started the assignment of $x = 3$. \\
  Second, some other processes which are just getting started assign $x=1$ and do not start the assignment of $x=2$.\\
  Third, The processes which are in the same places with first step assign $x=3$.\\\\
  The change in x is as follows:
  \begin{center}
    $x= ...\ 2\ 1\ ...\ 1\ 3\ ...$
  \end{center}

  LTL formula:
  \begin{center}
    $ \diamond\  (x = 2\ \mathcal{U}\ (x = 1 \ \mathcal{U}\ x = 3)) $\\
  $ \{<> (x == 2)\ until\ (x == 1\ until\ x == 3)\}$
  \end{center}
  \begin{quote}
    \\
  \end{quote}
  
\end{quote}
\end{answer}


\end{Question}

\end{document}