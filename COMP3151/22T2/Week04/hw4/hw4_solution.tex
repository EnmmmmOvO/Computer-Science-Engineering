\documentclass{article}

\usepackage{hyperref}

\usepackage{mathtools}
\usepackage{amssymb}
\usepackage{stmaryrd}

\usepackage{listings}

\usepackage{enumitem}
\usepackage[utf8]{inputenc}

\DeclareUnicodeCharacter{2227}{$\wedge$}
\DeclareUnicodeCharacter{2228}{$\vee$}

\usepackage{multicol}

\newcommand{\set}[1]{\{#1\}}
\DeclareMathOperator{\nat}{\mathbb{N}}

\begin{document}
\title{Homework 4 Commentary \\ COMP3151 T2 2021}
\author{Vincent Jackson}
\maketitle

\section{Smoker's Protocol}

In the smoker's protocol, we asked you to complete
a six agent resource distribution problem using semaphores.

The Smoker's protocol has two correctness conditions which you needed to meet.
The first correctness condition was that the protocol doesn't deadlock.

The second correctness condition is induced by the description of the problem:
each smoker requires resources they must procure from an agent,
which provides these resources.
Thus, the protocol must not produce resources out of thin air.

Another resource condition we might wish to have, but is less about correctness,
is that we don't waste resources;
e.g. that we don't pull three matches from the agents
for every one match that actually gets to a smoker.

I have attached a promela file which models the protocol, which you can use
to show deadlock freedom, and which also contains an LTL formula which ensures
that all resources come from somewhere, and a formula which ensures resources
aren't stockpiled without being used.

\section{Semaphores}

\subsection{Busy-wait Semaphore}

To begin, there was a good deal of confusion about \texttt{Thread.yield()}.
In Java, \texttt{Thread.yield()} is a hint to the scheduler that this
thread wants to yield. Importantly, Java \emph{can just ignore the hint}.
Thus, a good rule of thumb for using \texttt{Thread.yield()} is that your algorithm
must continue to work even if you replaced the yield with a no-op.

A second rule of thumb is that you should not loop inside a synchronised block
when the action that would end the loop is synchronised on the same lock.
The code in \autoref{deadlock_yield} deadlocks, because once \texttt{V} starts to loop, \texttt{P}
can never run and end the loop.

\begin{figure}
\begin{lstlisting}[language=java,basicstyle=\small\ttfamily]
  void synchronized P() {
    while (permits <= 0) {
      // Doesn't release the lock. As I said earlier,
      // replace Thread.yield with a no-op,
      // and the protocol still needs to work.
      Thread.yield();
    }
    permits--;
  }
  
  void synchronized V() {
    permits++;
  }
\end{lstlisting}
\caption{Deadlock due to Yield not Releasing the Lock}
\label{deadlock_yield}
\end{figure}

Here are two correct solutions. The first \autoref{vw_correct_1} puts all
the logic into \texttt{P} and \texttt{V}.
Another good solution \autoref{vw_correct_2} extracts the decrement logic into
a synchronised method.

\begin{figure}
\begin{lstlisting}[language=java,basicstyle=\small\ttfamily]
void P() {
  while (true) {
    // for better performance, check before actually trying
    while (permits <= 0) {
      Thread.yield();
    }
    synchronized(this) {
      if (permits > 0) {
        permits--;
        break;
      }
    }
  }
}

void synchronized V() {
  // synchronized because ++ is not necessarily atomic.
  permits++;
}
\end{lstlisting}
\caption{Very-Weak Semaphore Solution \#1}
\label{vw_correct_1}
\end{figure}

\begin{figure}
\begin{lstlisting}[language=java,basicstyle=\small\ttfamily]
  bool synchronized testAndDecr() {
    if permits > 0 {
      permits--;
      return true;
    } else {
      return false;
    }
  }

  void P() {
    while (!testAndDecr()) {
      Thread.yield()
    }
  }
  
  void synchronized V() {
    permits++;
  }
\end{lstlisting}
\caption{Very-Weak Semaphore Solution \#2}
\label{vw_correct_2}
\end{figure}


\subsection{Weak Semaphore}

This was well done by most students. As wait \emph{does} release the lock
when called, we can safely use a loop inside the synchronised block.

An important point to make is that Java monitors behave slightly differently
to what you might expect. With a classic monitor \cite{HoareMonitors},
notifying a waiting process will immediately wake up that process.
Thus the solution in \autoref{w_classical} works.
However, this is not the case in Java.

In Java, processes waiting on a monitor are not prioritised over
other threads. Thus after waking up,
the process \emph{must} check the guard again, as in \autoref{w_correct}.

\begin{figure}
  \begin{lstlisting}[language=java,basicstyle=\small\ttfamily]
    void synchronized P() {
      if (permits < 0) {
        wait();
      }
      permits--;
    }

    void synchronized V() {
      permits++;
      notify();
    }
  \end{lstlisting}
  \caption{Classical Weak Semaphore Solution (Doesn't work in Java)}
  \label{w_classical}
\end{figure}

\begin{figure}
  \begin{lstlisting}[language=java,basicstyle=\small\ttfamily]
    void synchronized P() {
      while (permits < 0) {
        wait();
      }
      permits--;
    }

    void synchronized V() {
      permits++;
      notify();
    }
  \end{lstlisting}
  \caption{Weak Semaphore Solution}
  \label{w_correct}
\end{figure}

\subsection{Strong Semaphore}

Strong semaphores require processes which call \texttt{P}
are released in the order in which they called \texttt{P}.

Because Java doesn't prioritise processes that have been signalled to
end their \texttt{wait}, we can't assume the processes in the queue are
prioritised over those which have just called \texttt{P}.
Thus in this solution, every process must be entered into the queue,
regardless of whether \texttt{permits > 0}.

\begin{figure}
  \begin{lstlisting}[language=java,basicstyle=\small\ttfamily]
    // Use whichever Java queue implementation you want
    Queue<long> queue = ...;

    void synchronized P() {
      queue.add(this.getId());
      while (permits < 0 || queue.peek() != this.getId()) {
        wait();
      }
      queue.poll();
      permits--;
    }

    void synchronized V() {
      permits++;
      notifyAll();
    }
  \end{lstlisting}
  \caption{Strong Semaphore Solution}
  \label{s_correct}
\end{figure}

\begin{thebibliography}{9}

\bibitem{HoareMonitors}
C. A. R. Hoare. 1974.
Monitors: an operating system structuring concept.
Commun. ACM 17, 10 (Oct. 1974), 549–557.
DOI:\url{https://doi.org/10.1145/355620.361161}
\end{thebibliography}

\end{document}
