\documentclass{article}

\usepackage[printqrbox=false,printhint=false,printanswer=true,printmarkingguide=false,printdraftpaper=false]{unswalgos}

\usepackage{tikz}
\usetikzlibrary{patterns}
\usetikzlibrary{shapes,fit}
\usepackage{tkz-fct}
\usepackage{wrapfig}
\usepackage{subfig}

\usepackage{mathtools}
\usepackage{amssymb}
\usepackage{booktabs,multicol,multirow}
\usepackage{wasysym}
\usepackage{tcolorbox}

\DeclareMathOperator*{\argmax}{arg\,max}
\DeclareMathOperator*{\argmin}{arg\,min}
\DeclareMathOperator{\NAND}{NAND}
\DeclareMathOperator{\AND}{AND}
\DeclareMathOperator{\OR}{OR}
\DeclareMathOperator{\NOT}{NOT}

\usepackage{xspace}

\fancyfoot[L]{\leftmark}
\fancyfoot[R]{\rightmark}

% This enables new paragraphs without indentation
\usepackage[parfill]{parskip}

\newcommand{\sem}{22T2}
\newcommand{\semester}{Term 2, 2022}
\SubjectNo{COMP3121/9101}
\newcommand{\taskname}{Assignment 1, Question 4}
\Institution{Jinghan Wang, z5286124} % Replace this with your name and zID


\begin{document}

\setcounter{question}{3}

\begin{Question}
Your friend has constructed an array $A$ of $n$ distinct integers, where $n \ge 2$. However, you cannot access the elements of the array directly; instead, they instead only allow you to ask questions of the form ``What is the maximum value amongst $A[l], A[l+1], \ldots, A[r-1], A[r]$?", where you may choose any valid indices $l$ and $r$ such that $l \le r$. You may assume any questions you ask are answered in constant time.

Your goal is to determine the value of the \textbf{second largest} element in the array.

\begin{Subquestion}
\textbf{[2 marks]} How can you find the value of the largest element in the array using only one question?

\begin{answer}
Answer:
\begin{quote}
    What is the maximum value amongst $A[\ell..r]$?\\
\end{quote}
\end{answer}
\end{Subquestion}

\begin{Subquestion}
\textbf{[2 marks]} If you know that the largest element occurs at index $i$, how could you then find the value of the second largest element using only two questions?

\begin{answer}
Answer:
\begin{quote}
1. What is the maximum value amongst $A[\ell..(i-1)]$?\\
2. What is the maximum value amongst $A[(i+1)..r]$?\\\\
The second largest of array is $max(A[\ell..(i-1)], A[(i+1)..r])$\\
\end{quote}
\end{answer}
\end{Subquestion}

\clearpage
\begin{Subquestion}\label{second-largest}
\textbf{[11 marks]} Design an algorithm which runs in $O(\log n)$ time and determines the value of the second largest element in the array.

\begin{answer}
Answer:
\begin{quote}
Get the maximum value amongst $A[\ell..r]$, set $max = A[\ell..r]$, the range of array is $[\ell..n]$, the index of maximum value is p.
\begin{quote}
- If $\ell + 1 = r$, \\
- $\quad \max(A[r..r]) == max$, the largest index is $p = r$.\\
- $\quad \max(A[\ell..\ell]) == max$, the largest index is $p = \ell$.\\
- If $\ell = r$, the largest index is $p = \ell$.\\
- Else\\
- $\quad$ Suppose $n = \displaystyle{\lfloor \frac{\ell + r}{2}  \rfloor} $\\
- $\quad$ Get the maximum value of left part $\max(A[\ell..n])$ and right part $\max(A[n..r])$. As the array constructed by distinct integers, $\max(A[\ell..n]) \neq \max(A[n..r])$\\
- $\quad$ If $\max(A[n..r]) == max$, the range change to $[n..r]$, and and then go back and start the next comparison.\\
- $\quad$ If $\max(A[\ell..n]) == max$, the range change to $[\ell..n]$, and and then go back and start the next comparison.\\\\
After getting the largest index, use 4.2 method, the second largest value is \\$max(A[\ell..(p-1)], A[(p+1)..r])$.
\end{quote}
Reason:\\
Use dichotomy to find the index corresponding to the largest number, and then compare the data on both sides of the largest number to obtain the second largest number.\\
\end{quote}
\end{answer}
\end{Subquestion}

\begin{Subquestion}\label{second-largest-ii}
\textbf{[5 marks]} Now your friend imposes an extra restriction: for each question you ask except the first, the value of $r-l$ should be no larger than the value of $r-l$ in the previous question. Subject to this restriction, design an algorithm which runs in $O(\log n)$ time and determines the value of the second largest element in the array.

You may choose to skip 4.3, in which case we will mark your submission for 4.4 as if it was submitted for 4.3 also.

\begin{answer}
Your answer here.
\end{answer}
\end{Subquestion}
\end{Question}

\end{document}