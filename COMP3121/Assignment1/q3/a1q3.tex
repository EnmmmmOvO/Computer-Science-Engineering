\documentclass{article}

\usepackage[printqrbox=false,printhint=false,printanswer=true,printmarkingguide=false,printdraftpaper=false]{unswalgos}

\usepackage{tikz}
\usetikzlibrary{patterns}
\usetikzlibrary{shapes,fit}
\usepackage{tkz-fct}
\usepackage{wrapfig}
\usepackage{subfig}

\usepackage{mathtools}
\usepackage{amssymb}
\usepackage{booktabs,multicol,multirow}
\usepackage{wasysym}
\usepackage{tcolorbox}

\DeclareMathOperator*{\argmax}{arg\,max}
\DeclareMathOperator*{\argmin}{arg\,min}
\DeclareMathOperator{\NAND}{NAND}
\DeclareMathOperator{\AND}{AND}
\DeclareMathOperator{\OR}{OR}
\DeclareMathOperator{\NOT}{NOT}

\usepackage{xspace}

\fancyfoot[L]{\leftmark}
\fancyfoot[R]{\rightmark}

% This enables new paragraphs without indentation
\usepackage[parfill]{parskip}

\newcommand{\sem}{22T2}
\newcommand{\semester}{Term 2, 2022}
\SubjectNo{COMP3121/9101}
\newcommand{\taskname}{Assignment 1, Question 3}
\Institution{Jinghan Wang, z5286124} % Replace this with your name and zID


\begin{document}

\setcounter{question}{2}

\begin{Question}
You are given an array $A$ containing each integer from $1$ to $n$ exactly once. Your task is to compute $f(A)$, the sum of $\max(A[\ell..r])$ over all pairs of indices $(\ell,r)$ such that $\ell \le r$.

\begin{Subquestion}
\textbf{[2 marks]} Suppose $n = 3$ and the array is $A = [2,1,3]$. Determine the value of $f(A)$.

\begin{answer}
Answer: $f(A) = 14$\\
Proof:
\begin{quote}
    According to the topic, the subarray of A is 
    \begin{center}
        $[2]$, $[1]$, $[3]$, $[2, 1]$, $[2, 3]$, $[1, 3]$, $[2, 1, 3]$
    \end{center}
    But $A[\ell..r]$ means that the element must be continued. Therefore, $f(A)$ will find between
    \begin{center}
        $[2]$, $[1]$, $[3]$, $[2, 1]$, $[2, 3]$, $[2, 1, 3]$
    \end{center}
    $f(A) = 2 + 1 + 3 + 2 + 3 + 3 = 14$.\\
\end{quote}
\end{answer}
\end{Subquestion}

For 3.2 and 3.3, suppose $i$ and $j$ are indices such that $1 \le i \le j \le n$, and let $g(i,j)$ be the number of subarrays $A[\ell..r]$ where $r > j$ and the maximum value is $A[i]$.

\begin{Subquestion}\label{identify-max}
\textbf{[4 marks]} For a given pair of indices $(i,j)$, under what conditions is $g(i,j)$ nonzero? In other words, what is the criterion for $A[i]$ to be the maximum of some subarray with its right endpoint at an index greater than $j$?

\begin{answer}
Answer:
\begin{quote}
    According to th topic, as long as it is not sorted from small to large, there is a value larger than the adjacent to the right of the value. It \\
\end{quote}
\end{answer}
\end{Subquestion}

\begin{Subquestion}\label{count-max}
\textbf{[6 marks]} Given an index $j$, design an algorithm which runs in $O(n)$ time and determines the values of $g(i,j)$ for all $i < j$.

\begin{answer}
Answer:
\begin{quote}
    
\end{quote}
\end{answer}
\end{Subquestion}

\begin{Subquestion}
\textbf{[8 marks]} Design an algorithm which runs in $O(n \log n)$ time and determines the value of $f(A)$.

\begin{answer}
Your answer here.
\end{answer}
\end{Subquestion}
\end{Question}

\end{document}